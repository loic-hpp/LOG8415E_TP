\documentclass[a4paper,11pt]{report}
\usepackage[utf8]{inputenc}
\usepackage{graphicx}
\usepackage{hyperref}
\usepackage{amsmath}
\usepackage{amsfonts}
\usepackage{geometry}
\geometry{margin=1in}
\usepackage{pdflscape}
\usepackage{fancyhdr}
\pagestyle{fancy}
\fancyhf{}
\fancyfoot[C]{\thepage}
\renewcommand{\headrulewidth}{0pt}
\renewcommand{\footrulewidth}{0pt}

\usepackage{listings}
\lstset{
    basicstyle=\ttfamily\small,
    breaklines=true
}

\usepackage{titling}
\pretitle{%
    \begin{center}
    \includegraphics[width=0.7\textwidth]{images/Logo_Polytechnique_Montréal.png}\\[\bigskipamount]
    \vspace*{1cm}
}
\posttitle{\end{center}}
\title{
\Large LOG8415E -- Advanced Concepts of Cloud Computing\\[1.5ex]
\large Final Assignment\\[1.5ex]
\large \textbf{Cloud Design Patterns: Implementing a DB Cluster}
\vspace*{2cm}
}
\author{
Loic Nguemegne Temena - 2180845
\vspace*{1cm}\\[2ex]
\vspace*{0.25cm}
Submitted to\\
Vahid Majdinasab
\vspace*{4cm}
}
\date{Fall 2025 -- 16 december 2025}

\begin{document}

\maketitle

\tableofcontents

\chapter{Benchmarking MySQL with sysbench}
Sysbench is a tool design for simulating realistic MySQL workloads. It can be used to 
benchmark and stress test MySQL databases by generating a variety of workloads, 
including read-only, write-only, and mixed read/write operations. 
Sysbench allows users to configure various parameters such as the number of threads, 
the size of the database, and the type of workload to simulate different scenarios 
and measure performance metrics like transactions per second (TPS), latency, and throughput.
In this assignment we use it to benchmark our database by running the following command:
\begin{lstlisting}[language=bash, frame=single]
sudo sysbench /usr/share/sysbench/oltp_read_only.lua --mysql-db=sakila --mysql-user="$MYSQL_USER" --mysql-password="$MYSQL_PASSWORD" prepare
sudo sysbench /usr/share/sysbench/oltp_read_only.lua --mysql-db=sakila --mysql-user="$MYSQL_USER" --mysql-password="$MYSQL_PASSWORD" run
sudo sysbench /usr/share/sysbench/oltp_read_only.lua --mysql-db=sakila --mysql-user="$MYSQL_USER" --mysql-password="$MYSQL_PASSWORD" cleanup
\end{lstlisting}
In figure~\ref{fig:sysbench_results}, we can see the benchmark results with a low latency of 
\textbf{1.90} ms and a throughput of \textbf{524.18} transactions per second (TPS) on average.
\begin{figure}
    \centering
    \includegraphics[width=0.8\textwidth]{images/Sysbenc_results.png}
    \caption{Sysbench results.}
    \label{fig:sysbench_results}
\end{figure}
\thispagestyle{fancy}
\chapter{Implementation of The Proxy pattern}
The proxy is running on an EC2 instance called \texttt{Proxy} in our architecture. 
It is located in the private subnet, it only accept request from the public subnet \textbf{10.0.0.0/16}
and is responsible for routing the incoming database queries to the appropriate database instance 
(master or worker) based on the type of query (read or write). The proxy uses FastAPI\cite{fastapi2021} to 
handle incoming HTTP requests and forwards them to the correct database instance using the 
\texttt{requests} library. the request is extract from the json payload by getting value of the key \textbf{"query"}. There are 3 routes defined in the proxy:
\begin{itemize}
    \item \textbf{/directHit}: All requests are forwarded to the master database instance.
    \item \textbf{/randomHit}: Write queries are forwarded to the master database instance,
    while read queries are forwarded to a randomly selected worker database instance or the master.
    \item \textbf{/customHit}: Write queries are forwarded to the master database instance,
    while read queries are forwarded to the worker database instance with the lowest average response time.
    A thread ping each worker every 100 ms to measure their response time.
\end{itemize}
Each route call the 2 following functions:
\begin{lstlisting}[language=python, frame=single]
def connect_to_db(host: str):
    return pymysql.connect(
        host=host,
        user=MYSQL_USER,
        password=MYSQL_PASSWORD,
        database="sakila",
        ssl_disabled=True,
    )

def execute_query_on_worker(worker_ip: str, query: str):
    mysql_conn = connect_to_db(worker_ip)
    cursor = mysql_conn.cursor()
    cursor.execute(query)
    results = cursor.fetchall()
    if query.startswith(("UPDATE", "update", "INSERT", "insert")):
        mysql_conn.commit()
        results = "Update query executed successfully"
    cursor.close()
    mysql_conn.close()
    return results
\end{lstlisting}
\chapter{Implementation of The Gatekeeper pattern}
The gatekeeper is running on an EC2 instance called \texttt{Gatekeeper} in our architecture. 
It is located in the public subnet, it accepts request from the internet and has the following responsibilities:
\begin{itemize}
    \item \textbf{Authenticate} Authenticate incoming requests to ensure they are from trusted sources. Authentication is 
    performed by checking the key \texttt{auth\_key}, which must have the value \texttt{tp3-log8415e}. 
    If authentication fails, the gatekeeper returns a \texttt{401 Unauthorized} response.
    
    \item \textbf{Confirm the query presence} Validate that incoming requests contain a query. 
    The query is extracted from the JSON payload by retrieving the value of the key \texttt{query}. 
    If the query is missing, the gatekeeper returns a \texttt{400 Bad Request} response.

    \item \textbf{Block malicious queries} Prevent potentially harmful queries from reaching the database.
    The gatekeeper checks if the query contains any of the following keywords: \texttt{DROP}, \texttt{DELETE}, \texttt{ALTER}, \texttt{TRUNCATE}. 
    If any of these keywords are found in the query, the gatekeeper returns a \texttt{403 Forbidden} response.
    The only request accepted are read and write queries (\texttt{SELECT}, \texttt{INSERT}, \texttt{UPDATE}).
    
    \item \textbf{Forward requests} Forward authenticated requests to the proxy server located in the private subnet.
    All the route will call the function \texttt{handle\_hit} to handle the requests.
\end{itemize}


\chapter{Benchmarking the cluster}
The benchmarking of the implemented database cluster is performed via the file \texttt{benchmark.py}.
to use the script you need to provide the public ip address of the gatekeeper server as argument and 
run the following command:
\begin{lstlisting}[language=bash, frame=single]
python benchmark.py GATEKEEPER_PUBLIC_IP
\end{lstlisting}
The script will send a total of 2000 requests (1000 read and 1000 write) to the gatekeeper server, 
for each of the 3 routes (\texttt{directHit}, \texttt{randomHit}, \texttt{customHit}),
measuring the response time for each request and calculating the average response time for read and write operations separately.
To perform the requests asynchronously, the script uses the \texttt{aiohttp} and \texttt{asyncio} libraries.
The results of the benchmarking are printed to the console, showing the average response time for read and write operations in milliseconds.

\chapter{Describe clearly how your implementation works.}
My implementation is composed of the following components:
\begin{itemize}
    \item \textbf{VPC}\cite{aws_vpc} : A Virtual Private Cloud (VPC) that provides an isolated network environment for the database cluster.
    \item \textbf{Public subnet}: A subnet within the VPC that is accessible from the internet. It contains the gatekeeper server and the NAT.
    \item \textbf{Private subnet}: A subnet within the VPC that is not accessible from the internet. It contains the proxy server and database instances.
    \item \textbf{NAT gateway}: A Network Address Translation (NAT) gateway that allows instances in the private subnet to access the internet.
    \item \textbf{Gatekeeper server}: An application running on an EC2 instance\cite{aws_ec2} in the public subnet that 
    authenticates and validates incoming requests before forwarding them to the proxy server.

    \item \textbf{Proxy server}: An application running on an EC2 instance in the private subnet that routes 
    incoming database queries to the appropriate database instance (master or worker) based on the type of query (read or write) and the route.
    
    \item \textbf{Master database instance}: A MySQL database instance that handles all write operations and 
    can also handle read operations. It is located on an EC2 instance in the private subnet.

    \item \textbf{2 Worker database instances}: Multiple MySQL database instances that handle read operations only. 
    They are located on EC2 instances in the private subnet.
    
\end{itemize}

The figure~\ref{fig:architecture} illustrates the overall architecture of the implemented database cluster. The solution allows users to 
perform read and write operations on the database while ensuring security and efficient query routing. if a request is completed
successfully, it will return in the response body the worker\_ip, the query and the results.
And if the request fails, it will return an error message with the corresponding status code.

to use the service you should send a post request on the route \\ http://GATEKEEPER\_PUBLIC\_IP:8000/{route} with a json payload like this:
\begin{lstlisting}[language=bash, frame=single]
    {
        "auth_key": "tp3-log8415e",
        "query": "SELECT * FROM actor LIMIT 5;"
    }
\end{lstlisting}
Where \texttt{\{route\}} can be \texttt{directHit}, \texttt{randomHit} or \texttt{customHit} depending on the desired routing strategy.
To run the services on the gatekeeper and the proxy we use \texttt{uvicorn}\cite{uvicorn2024}.


\begin{figure}
    \centering
    \includegraphics[width=0.8\textwidth]{images/architecture.png}
    \caption{Architecture.}
    \label{fig:architecture}
\end{figure}

\chapter{Summary of results }
The benchmarking results of the implemented database cluster are shown 
in figure~\ref{fig:benchmarking_results}. The results indicate that the read operations provide the best performance, 
with a response time between \textbf{5.71 ms per request} and \textbf{5.91 ms per request}.
Write operations have a higher response time, ranging from \textbf{11.71 ms per request} to \textbf{11.90 ms per request}.

Also the figure~\ref{fig:cpu_utilization} shows the CPU utilization of the proxy server during the benchmarking process. 
It shows that the proxy has the highest CPU utilization which i expected because it does more processing than any other instance.
And the Gatekeeper has the lowest CPU utilization since it only authenticate and validate the requests before forwarding them to the proxy.

\begin{figure}
    \centering
    \includegraphics[width=1.0\textwidth]{images/cpu_utilization_when_benchmarking.png}
    \caption{CPU utilization when benchmarking.}
    \label{fig:cpu_utilization}
\end{figure}

\begin{figure}
    \centering
    \includegraphics[width=0.8\textwidth]{images/benchmarking_results.png}
    \caption{Benchmarking results.}
    \label{fig:benchmarking_results}
\end{figure}
\chapter{Instructions to run your code}
Instructions to run the project code are provided in the project \texttt{README.md} file. Please refer to the \texttt{README.md} for detailed setup and execution steps.
\bibliographystyle{plain}
\bibliography{references}

\end{document}